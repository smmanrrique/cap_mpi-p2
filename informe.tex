% Options for packages loaded elsewhere
\PassOptionsToPackage{unicode}{hyperref}
\PassOptionsToPackage{hyphens}{url}
%
\documentclass[
]{article}
\usepackage{lmodern}
\usepackage{amsmath}
\usepackage{ifxetex,ifluatex}
\ifnum 0\ifxetex 1\fi\ifluatex 1\fi=0 % if pdftex
  \usepackage[T1]{fontenc}
  \usepackage[utf8]{inputenc}
  \usepackage{textcomp} % provide euro and other symbols
  \usepackage{amssymb}
\else % if luatex or xetex
  \usepackage{unicode-math}
  \defaultfontfeatures{Scale=MatchLowercase}
  \defaultfontfeatures[\rmfamily]{Ligatures=TeX,Scale=1}
\fi
% Use upquote if available, for straight quotes in verbatim environments
\IfFileExists{upquote.sty}{\usepackage{upquote}}{}
\IfFileExists{microtype.sty}{% use microtype if available
  \usepackage[]{microtype}
  \UseMicrotypeSet[protrusion]{basicmath} % disable protrusion for tt fonts
}{}
\makeatletter
\@ifundefined{KOMAClassName}{% if non-KOMA class
  \IfFileExists{parskip.sty}{%
    \usepackage{parskip}
  }{% else
    \setlength{\parindent}{0pt}
    \setlength{\parskip}{6pt plus 2pt minus 1pt}}
}{% if KOMA class
  \KOMAoptions{parskip=half}}
\makeatother
\usepackage{xcolor}
\IfFileExists{xurl.sty}{\usepackage{xurl}}{} % add URL line breaks if available
\IfFileExists{bookmark.sty}{\usepackage{bookmark}}{\usepackage{hyperref}}
\hypersetup{
  pdftitle={Práctica 2: Programación de Comunicaciones en MPI},
  pdfauthor={Shamuel Manrrique 802400 Aldrix Marfil 794976},
  hidelinks,
  pdfcreator={LaTeX via pandoc}}
\urlstyle{same} % disable monospaced font for URLs
\usepackage[margin=1in]{geometry}
\usepackage{color}
\usepackage{fancyvrb}
\newcommand{\VerbBar}{|}
\newcommand{\VERB}{\Verb[commandchars=\\\{\}]}
\DefineVerbatimEnvironment{Highlighting}{Verbatim}{commandchars=\\\{\}}
% Add ',fontsize=\small' for more characters per line
\usepackage{framed}
\definecolor{shadecolor}{RGB}{248,248,248}
\newenvironment{Shaded}{\begin{snugshade}}{\end{snugshade}}
\newcommand{\AlertTok}[1]{\textcolor[rgb]{0.94,0.16,0.16}{#1}}
\newcommand{\AnnotationTok}[1]{\textcolor[rgb]{0.56,0.35,0.01}{\textbf{\textit{#1}}}}
\newcommand{\AttributeTok}[1]{\textcolor[rgb]{0.77,0.63,0.00}{#1}}
\newcommand{\BaseNTok}[1]{\textcolor[rgb]{0.00,0.00,0.81}{#1}}
\newcommand{\BuiltInTok}[1]{#1}
\newcommand{\CharTok}[1]{\textcolor[rgb]{0.31,0.60,0.02}{#1}}
\newcommand{\CommentTok}[1]{\textcolor[rgb]{0.56,0.35,0.01}{\textit{#1}}}
\newcommand{\CommentVarTok}[1]{\textcolor[rgb]{0.56,0.35,0.01}{\textbf{\textit{#1}}}}
\newcommand{\ConstantTok}[1]{\textcolor[rgb]{0.00,0.00,0.00}{#1}}
\newcommand{\ControlFlowTok}[1]{\textcolor[rgb]{0.13,0.29,0.53}{\textbf{#1}}}
\newcommand{\DataTypeTok}[1]{\textcolor[rgb]{0.13,0.29,0.53}{#1}}
\newcommand{\DecValTok}[1]{\textcolor[rgb]{0.00,0.00,0.81}{#1}}
\newcommand{\DocumentationTok}[1]{\textcolor[rgb]{0.56,0.35,0.01}{\textbf{\textit{#1}}}}
\newcommand{\ErrorTok}[1]{\textcolor[rgb]{0.64,0.00,0.00}{\textbf{#1}}}
\newcommand{\ExtensionTok}[1]{#1}
\newcommand{\FloatTok}[1]{\textcolor[rgb]{0.00,0.00,0.81}{#1}}
\newcommand{\FunctionTok}[1]{\textcolor[rgb]{0.00,0.00,0.00}{#1}}
\newcommand{\ImportTok}[1]{#1}
\newcommand{\InformationTok}[1]{\textcolor[rgb]{0.56,0.35,0.01}{\textbf{\textit{#1}}}}
\newcommand{\KeywordTok}[1]{\textcolor[rgb]{0.13,0.29,0.53}{\textbf{#1}}}
\newcommand{\NormalTok}[1]{#1}
\newcommand{\OperatorTok}[1]{\textcolor[rgb]{0.81,0.36,0.00}{\textbf{#1}}}
\newcommand{\OtherTok}[1]{\textcolor[rgb]{0.56,0.35,0.01}{#1}}
\newcommand{\PreprocessorTok}[1]{\textcolor[rgb]{0.56,0.35,0.01}{\textit{#1}}}
\newcommand{\RegionMarkerTok}[1]{#1}
\newcommand{\SpecialCharTok}[1]{\textcolor[rgb]{0.00,0.00,0.00}{#1}}
\newcommand{\SpecialStringTok}[1]{\textcolor[rgb]{0.31,0.60,0.02}{#1}}
\newcommand{\StringTok}[1]{\textcolor[rgb]{0.31,0.60,0.02}{#1}}
\newcommand{\VariableTok}[1]{\textcolor[rgb]{0.00,0.00,0.00}{#1}}
\newcommand{\VerbatimStringTok}[1]{\textcolor[rgb]{0.31,0.60,0.02}{#1}}
\newcommand{\WarningTok}[1]{\textcolor[rgb]{0.56,0.35,0.01}{\textbf{\textit{#1}}}}
\usepackage{graphicx}
\makeatletter
\def\maxwidth{\ifdim\Gin@nat@width>\linewidth\linewidth\else\Gin@nat@width\fi}
\def\maxheight{\ifdim\Gin@nat@height>\textheight\textheight\else\Gin@nat@height\fi}
\makeatother
% Scale images if necessary, so that they will not overflow the page
% margins by default, and it is still possible to overwrite the defaults
% using explicit options in \includegraphics[width, height, ...]{}
\setkeys{Gin}{width=\maxwidth,height=\maxheight,keepaspectratio}
% Set default figure placement to htbp
\makeatletter
\def\fps@figure{htbp}
\makeatother
\setlength{\emergencystretch}{3em} % prevent overfull lines
\providecommand{\tightlist}{%
  \setlength{\itemsep}{0pt}\setlength{\parskip}{0pt}}
\setcounter{secnumdepth}{-\maxdimen} % remove section numbering
\ifluatex
  \usepackage{selnolig}  % disable illegal ligatures
\fi

\title{Práctica 2: Programación de Comunicaciones en MPI}
\author{Shamuel Manrrique 802400 \n Aldrix Marfil 794976}
\date{08/01/2021}

\begin{document}
\maketitle

\hypertarget{comunicaciones-en-mpi}{%
\section{Comunicaciones en MPI}\label{comunicaciones-en-mpi}}

En esta práctica se proponen algunos patrones de comunicación clásicos
en paso de mensajes. Se pide realizar dos tareas con estos patrones:

\begin{itemize}
\item
  La primera tarea consiste en la implementación de estos patrones en
  lenguaje C utilizando la librería de MPI disponible
\item
  La segunda tarea consiste en la caracterización del comportamiento de
  estos patrones de comunicación
\end{itemize}

\hypertarget{implementaciuxf3n-en-c-de-comunicaciones-en-mpi}{%
\subsection{Implementación en C de comunicaciones en
MPI}\label{implementaciuxf3n-en-c-de-comunicaciones-en-mpi}}

Para lograr los requerimientos de esta práctica se realizaron tres
implementaciones:

\begin{itemize}
\item
  Test de latencia: Para esta sección se implementó el script
  \textbf{latency\_test.c} el cual es un programa en MPI que mide la
  latencia de las comunicaciones por pareja asignadas a jugar
  ping-pong(rebotar paquetes) con un número parametrizado de envíos.
\item
  Test de ancho de banda: Para esta sección se implementó el script
  \textbf{bandwidth\_test.c} el cual es un programa en MPI que mide el
  ancho de bandas de un envío de N paquetes de tamaño M entre dos
  procesos un emisor y un receptor que emite un ACK de validación una
  vez recibido los N paquetes.
\item
  Test de latencia de la operación de broadcast: Para esta sección se
  implementó el script \textbf{broadcast\_test.c} el cual es un programa
  en MPI que tiene dos funcionalidades:

  \begin{enumerate}
  \def\labelenumi{\arabic{enumi}.}
  \item
    Broadcast One-to-one: Envio de mensajes usando Send/Receive en donde
    un proceso es el root y tiene que hacer el envió uno a uno al resto
    de procesos.
  \item
    Broadcast Bcast: Envío usando la función de Bcast de MPI el cual
    realiza el envío simultáneo a todos los procesos, solo se le indica
    el proceso raíz en la misma llamada.
  \end{enumerate}
\end{itemize}

\hypertarget{archivos-de-ejecuciuxf3n}{%
\subsection{Archivos de ejecución}\label{archivos-de-ejecuciuxf3n}}

Para la ejecución de forma automatizada y parametrizada de cada uno de
los script de c se creó un bash script para cada uno con los siguientes
valores especificados para la ejecución:

\begin{verbatim}
PROGRAM="name.out"         -> Nombre del archivo compilado con mpicc.
CSV_NAME="name_test"       -> Nombre del archivo csv con los resultados.
HOSTFILE="host_name.txt"   -> Nombre del Host File a usar
PACKET_SIZES=(500 1000)    -> Arreglo con los distintos tamaños del Paquete
NUMBER_PACKETS=(1000 5000) -> Arreglo con los distintos número de test/ejecuciones.
NUMBER_PROCCESS=(8 16)     -> Arreglo con los distintos número de procesos.
\end{verbatim}

Los Números usados en PACKET\_SIZES, NUMBER\_PACKETS, NUMBER\_PROCCESS
son simplemente para ejemplificar en los bash usados para los distintos
test estos valores son bastante grandes para poder apreciar con mejor
detalle las ventajas de la paralelización.

\hypertarget{archivo-resultante-de-los-programas-bash}{%
\subsection{Archivo Resultante de los programas
bash}\label{archivo-resultante-de-los-programas-bash}}

Para facilitar el análisis de los resultados obtenidos de tiempo de
ejecución, velocidad de transferencia y latencia de comunicación por
cada uno de los scripts se guarda un archivo de salida .csv con las
siguientes cabeceras:

\begin{verbatim}
BCAST_TYPE   -> Type de broadcast en caso de aplicar.
PACKET_SIZE  -> Tamaño en Bytes del paquete.
N_PACKETS    -> Cantidad de paquete a enviar.
N_BOUNCES    -> Número de rebotes del paquete de un proceso a otro.
NODE         -> Máquina donde se ejecuta el proceso.
PROCESS      -> Identificador(Rank) del proceso.
SRC          -> Proceso que envía el/los mensaje/s
DST          -> Proceso que recibe el/los mensaje/s
TAG          -> Identificador único de parejas conectadas
COM_TIME     -> Uso MPI_Wtime para medir el tiempo de envío y confirmación de recepción.
RUNING_TIME  -> Tiempo de ejecución del proceso.
\end{verbatim}

Por mantener consistencia en todos los .csv dependiendo del script que
se ejecute puede arrojar alguna columna vacía dado que ese dato no es
requerido para ese script en particular. Por otra parte el PACKET\_SIZE
como se está usando tipo entero (2 Bytes = 16 bits) se multiplica el
tamaño introducido por dieciséis bits para obtener la cantidad total en
bits, las medidas de tiempo usadas COM\_TIME y RUNING\_TIME estan
expresadas en segundos.

\hypertarget{anuxe1lisis-de-resultados-de-rendimiento}{%
\section{Análisis de resultados de
rendimiento}\label{anuxe1lisis-de-resultados-de-rendimiento}}

\hypertarget{importar-libreruxedas-necesarias}{%
\subsection{Importar librerías
necesarias}\label{importar-libreruxedas-necesarias}}

\begin{Shaded}
\begin{Highlighting}[]
\FunctionTok{library}\NormalTok{(}\StringTok{"ggplot2"}\NormalTok{)}
\FunctionTok{library}\NormalTok{(}\StringTok{"dplyr"}\NormalTok{)         }\CommentTok{\# load}
\FunctionTok{library}\NormalTok{(}\StringTok{"RcmdrMisc"}\NormalTok{)}
\FunctionTok{library}\NormalTok{(}\StringTok{"nleqslv"}\NormalTok{)       }\CommentTok{\# Resolver sistema ecuaciones lineales/no lineales}
\FunctionTok{library}\NormalTok{(}\StringTok{"readr"}\NormalTok{)}
\NormalTok{path }\OtherTok{=} \StringTok{"C:/Users/smmanrrique/3D Objects/unizar/cap/cap\_mpi{-}p2/results/"}
\end{Highlighting}
\end{Shaded}

\hypertarget{test-de-latencia-entre-procesos-en-mpi}{%
\subsection{Test de latencia entre procesos en
MPI}\label{test-de-latencia-entre-procesos-en-mpi}}

\begin{Shaded}
\begin{Highlighting}[]
\CommentTok{\# File name}
\NormalTok{archivo }\OtherTok{=} \StringTok{"latency\_1000pkt.csv"}
\NormalTok{read\_csv }\OtherTok{=} \FunctionTok{paste}\NormalTok{(path,archivo , }\AttributeTok{sep=}\StringTok{""}\NormalTok{)}

\CommentTok{\# Read and import csv}
\NormalTok{latency }\OtherTok{\textless{}{-}} \FunctionTok{read.csv}\NormalTok{(}\StringTok{"C:/Users/smmanrrique/3D Objects/unizar/cap/cap\_mpi{-}p2/results/latency\_1000pkt.csv"}\NormalTok{)}


\FunctionTok{numSummary}\NormalTok{(latency[,}\FunctionTok{c}\NormalTok{(}\StringTok{"COM\_TIME"}\NormalTok{), }\AttributeTok{drop=}\ConstantTok{FALSE}\NormalTok{], }\AttributeTok{groups =}\NormalTok{ latency}\SpecialCharTok{$}\NormalTok{NPROC, }\AttributeTok{statistics=}\FunctionTok{c}\NormalTok{(}\StringTok{"mean"}\NormalTok{, }\StringTok{"sd"}\NormalTok{, }\StringTok{"IQR"}\NormalTok{, }\StringTok{"quantiles"}\NormalTok{, }\StringTok{"skewness"}\NormalTok{, }\StringTok{"kurtosis"}\NormalTok{), }\AttributeTok{quantiles=}\FunctionTok{c}\NormalTok{(}\DecValTok{0}\NormalTok{,.}\DecValTok{25}\NormalTok{,.}\DecValTok{5}\NormalTok{,.}\DecValTok{75}\NormalTok{,}\DecValTok{1}\NormalTok{))}
\end{Highlighting}
\end{Shaded}

\begin{verbatim}
##            mean           sd     IQR skewness   kurtosis      0%      25%
## 4  2.746753e-06 9.062573e-05 0.0e+00 63.24895 4001.61266 1.0e-06 0.000001
## 8  2.073282e-04 6.839481e-05 3.8e-05 10.65340  312.16378 1.4e-05 0.000174
## 16 2.372200e-04 9.583415e-05 7.3e-05  5.06580   75.51097 1.4e-05 0.000192
##         50%      75%     100% COM_TIME:n
## 4  0.000001 0.000001 0.005735       4004
## 8  0.000199 0.000212 0.002502       8008
## 16 0.000212 0.000265 0.002437      16016
\end{verbatim}

\begin{Shaded}
\begin{Highlighting}[]
\FunctionTok{Boxplot}\NormalTok{(COM\_TIME}\SpecialCharTok{\textasciitilde{}}\NormalTok{NPROC, }\AttributeTok{data=}\NormalTok{latency, }\AttributeTok{id=}\FunctionTok{list}\NormalTok{(}\AttributeTok{method=}\StringTok{"y"}\NormalTok{))}
\end{Highlighting}
\end{Shaded}

\includegraphics{informe_files/figure-latex/unnamed-chunk-1-1.pdf}

\begin{verbatim}
##  [1] "18020" "16042" "16017" "18019" "16266" "16032" "16057" "18049" "18052"
## [10] "18824" "20041" "20376" "20416" "20172" "20425" "20442" "20102" "20264"
## [19] "20140" "20076" "21024" "21055" "20027" "20026" "20022" "20024" "20025"
## [28] "20023" "20021" "20028" "197"   "269"   "305"   "417"   "509"   "870"  
## [37] "871"   "969"   "1195"  "1558"  "11494" "11495" "408"   "8954"  "862"  
## [46] "8617"  "1728"  "175"   "8280"  "8379"
\end{verbatim}

\begin{Shaded}
\begin{Highlighting}[]
\NormalTok{sr\_latency }\OtherTok{=} \FunctionTok{filter}\NormalTok{(latency, TYPE}\SpecialCharTok{==}\StringTok{" SR"}\NormalTok{ )}
\FunctionTok{numSummary}\NormalTok{(sr\_latency[,}\FunctionTok{c}\NormalTok{(}\StringTok{"COM\_TIME"}\NormalTok{), }\AttributeTok{drop=}\ConstantTok{FALSE}\NormalTok{], }\AttributeTok{groups =}\NormalTok{ sr\_latency}\SpecialCharTok{$}\NormalTok{NPROC, }\AttributeTok{statistics=}\FunctionTok{c}\NormalTok{(}\StringTok{"mean"}\NormalTok{, }\StringTok{"sd"}\NormalTok{, }\StringTok{"IQR"}\NormalTok{, }\StringTok{"quantiles"}\NormalTok{, }\StringTok{"skewness"}\NormalTok{, }\StringTok{"kurtosis"}\NormalTok{), }\AttributeTok{quantiles=}\FunctionTok{c}\NormalTok{(}\DecValTok{0}\NormalTok{,.}\DecValTok{25}\NormalTok{,.}\DecValTok{5}\NormalTok{,.}\DecValTok{75}\NormalTok{,}\DecValTok{1}\NormalTok{))}
\end{Highlighting}
\end{Shaded}

\begin{verbatim}
##            mean           sd     IQR skewness   kurtosis       0%      25%
## 4  0.0000040750 1.282198e-04 0.0e+00 44.71271 1999.48230 0.000001 0.000001
## 8  0.0002086220 6.360127e-05 2.3e-05 12.98856  407.60025 0.000126 0.000187
## 16 0.0002383241 7.907545e-05 7.1e-05  4.60444   77.89572 0.000123 0.000194
##         50%      75%     100% COM_TIME:n
## 4  0.000001 0.000001 0.005735       2000
## 8  0.000200 0.000210 0.002472       4000
## 16 0.000212 0.000265 0.002405       8000
\end{verbatim}

\begin{Shaded}
\begin{Highlighting}[]
\CommentTok{\# }
\FunctionTok{Boxplot}\NormalTok{(COM\_TIME}\SpecialCharTok{\textasciitilde{}}\NormalTok{NPROC, }\AttributeTok{data=}\NormalTok{sr\_latency, }\AttributeTok{id=}\FunctionTok{list}\NormalTok{(}\AttributeTok{method=}\StringTok{"y"}\NormalTok{))}
\end{Highlighting}
\end{Shaded}

\includegraphics{informe_files/figure-latex/unnamed-chunk-1-2.pdf}

\begin{verbatim}
##  [1] "9001"  "8001"  "8016"  "9016"  "9402"  "9584"  "8561"  "9004"  "9036" 
## [10] "9072"  "13228" "13112" "13134" "12162" "10114" "11935" "12348" "10050"
## [19] "10121" "10756" "10510" "10004" "10003" "10001" "10002" "12721" "12720"
## [28] "12719" "10200" "11215" "5746"  "394"   "236"   "299"   "6"     "11"   
## [37] "3871"  "5"     "3"     "7"
\end{verbatim}

\begin{Shaded}
\begin{Highlighting}[]
\NormalTok{rs\_latency }\OtherTok{=} \FunctionTok{filter}\NormalTok{(latency, TYPE}\SpecialCharTok{==}\StringTok{" RS"}\NormalTok{ )}
\FunctionTok{numSummary}\NormalTok{(rs\_latency[,}\FunctionTok{c}\NormalTok{(}\StringTok{"COM\_TIME"}\NormalTok{), }\AttributeTok{drop=}\ConstantTok{FALSE}\NormalTok{], }\AttributeTok{groups =}\NormalTok{ rs\_latency}\SpecialCharTok{$}\NormalTok{NPROC, }\AttributeTok{statistics=}\FunctionTok{c}\NormalTok{(}\StringTok{"mean"}\NormalTok{, }\StringTok{"sd"}\NormalTok{, }\StringTok{"IQR"}\NormalTok{, }\StringTok{"quantiles"}\NormalTok{, }\StringTok{"skewness"}\NormalTok{, }\StringTok{"kurtosis"}\NormalTok{), }\AttributeTok{quantiles=}\FunctionTok{c}\NormalTok{(}\DecValTok{0}\NormalTok{,.}\DecValTok{25}\NormalTok{,.}\DecValTok{5}\NormalTok{,.}\DecValTok{75}\NormalTok{,}\DecValTok{1}\NormalTok{))}
\end{Highlighting}
\end{Shaded}

\begin{verbatim}
##            mean           sd     IQR  skewness  kurtosis      0%      25%
## 4  1.421158e-06 1.657453e-06 0.0e+00 11.586299 193.03602 1.0e-06 0.000001
## 8  2.060369e-04 7.285015e-05 5.0e-05  9.007644 248.60435 1.4e-05 0.000165
## 16 2.361180e-04 1.100377e-04 7.8e-05  4.996665  65.88757 1.4e-05 0.000187
##         50%      75%     100% COM_TIME:n
## 4  0.000001 0.000001 0.000039       2004
## 8  0.000198 0.000215 0.002502       4008
## 16 0.000213 0.000265 0.002437       8016
\end{verbatim}

\begin{Shaded}
\begin{Highlighting}[]
\FunctionTok{Boxplot}\NormalTok{(COM\_TIME}\SpecialCharTok{\textasciitilde{}}\NormalTok{NPROC, }\AttributeTok{data=}\NormalTok{rs\_latency, }\AttributeTok{id=}\FunctionTok{list}\NormalTok{(}\AttributeTok{method=}\StringTok{"y"}\NormalTok{))}
\end{Highlighting}
\end{Shaded}

\includegraphics{informe_files/figure-latex/unnamed-chunk-1-3.pdf}

\begin{verbatim}
##  [1] "8017"  "9019"  "8139"  "8032"  "9034"  "8577"  "9351"  "9420"  "9425" 
## [10] "9022"  "10031" "10206" "10226" "10104" "10230" "10239" "10069" "10150"
## [19] "10088" "10056" "10530" "10023" "10022" "10021" "10024" "13409" "10094"
## [28] "11352" "13575" "11897" "99"    "134"   "152"   "209"   "255"   "434"  
## [37] "435"   "484"   "597"   "771"   "5749"  "204"   "4466"  "430"   "4300" 
## [46] "854"   "88"    "4139"  "4187"  "616"
\end{verbatim}

\hypertarget{test-de-bandwidth-entre-procesos-en-mpi}{%
\subsection{Test de Bandwidth entre procesos en
MPI}\label{test-de-bandwidth-entre-procesos-en-mpi}}

\begin{Shaded}
\begin{Highlighting}[]
\CommentTok{\# File name}
\NormalTok{archivo }\OtherTok{=} \StringTok{"bandwidth\_1000pkt.csv"}
\NormalTok{read\_csv }\OtherTok{=} \FunctionTok{paste}\NormalTok{(path,archivo , }\AttributeTok{sep=}\StringTok{""}\NormalTok{)}

\CommentTok{\# Read and import csv}
\NormalTok{bandwidth }\OtherTok{\textless{}{-}} \FunctionTok{read.csv}\NormalTok{(read\_csv, }\AttributeTok{header=}\NormalTok{T, }\AttributeTok{dec=}\StringTok{\textquotesingle{}.\textquotesingle{}}\NormalTok{, }\AttributeTok{sep=}\StringTok{\textquotesingle{},\textquotesingle{}}\NormalTok{, }\AttributeTok{na.strings =} \StringTok{""}\NormalTok{)}

\FunctionTok{numSummary}\NormalTok{(bandwidth[,}\FunctionTok{c}\NormalTok{(}\StringTok{"COM\_TIME"}\NormalTok{), }\AttributeTok{drop=}\ConstantTok{FALSE}\NormalTok{], }\AttributeTok{groups =}\NormalTok{ bandwidth}\SpecialCharTok{$}\NormalTok{NPROC, }\AttributeTok{statistics=}\FunctionTok{c}\NormalTok{(}\StringTok{"mean"}\NormalTok{, }\StringTok{"sd"}\NormalTok{, }\StringTok{"IQR"}\NormalTok{, }\StringTok{"quantiles"}\NormalTok{, }\StringTok{"skewness"}\NormalTok{, }\StringTok{"kurtosis"}\NormalTok{), }\AttributeTok{quantiles=}\FunctionTok{c}\NormalTok{(}\DecValTok{0}\NormalTok{,.}\DecValTok{25}\NormalTok{,.}\DecValTok{5}\NormalTok{,.}\DecValTok{75}\NormalTok{,}\DecValTok{1}\NormalTok{))}
\end{Highlighting}
\end{Shaded}

\begin{verbatim}
##          mean           sd        IQR   skewness    kurtosis       0%
## 4  0.00250300 0.0002054832 0.00023000  1.2840598  0.87858497 0.002352
## 8  0.06600275 0.0062638009 0.00441200 -1.4033012 -0.04942453 0.055791
## 16 0.06790788 0.0017206640 0.00230475 -0.7220175 -0.01467492 0.064356
##           25%       50%        75%     100% COM_TIME:n
## 4  0.00235425 0.0024355 0.00258425 0.002789          4
## 8  0.06535825 0.0690700 0.06977025 0.070005          8
## 16 0.06701275 0.0679010 0.06931750 0.069966         16
\end{verbatim}

\begin{Shaded}
\begin{Highlighting}[]
\CommentTok{\# }
\FunctionTok{Boxplot}\NormalTok{(COM\_TIME}\SpecialCharTok{\textasciitilde{}}\NormalTok{NPROC, }\AttributeTok{data=}\NormalTok{bandwidth, }\AttributeTok{id=}\FunctionTok{list}\NormalTok{(}\AttributeTok{method=}\StringTok{"y"}\NormalTok{))}
\end{Highlighting}
\end{Shaded}

\includegraphics{informe_files/figure-latex/unnamed-chunk-2-1.pdf}

\begin{Shaded}
\begin{Highlighting}[]
\CommentTok{\#sqldf("SELECT  Región, COUNT(País) AS Count FROM Dataset GROUP BY Región order by count desc")}
\end{Highlighting}
\end{Shaded}

\hypertarget{test-de-broadcast-entre-procesos-en-mpi}{%
\subsection{Test de Broadcast entre procesos en
MPI}\label{test-de-broadcast-entre-procesos-en-mpi}}

\begin{Shaded}
\begin{Highlighting}[]
\CommentTok{\# File name}
\NormalTok{archivo }\OtherTok{=} \StringTok{"broadcast\_1000pkt.csv"}
\NormalTok{read\_csv }\OtherTok{=} \FunctionTok{paste}\NormalTok{(path,archivo , }\AttributeTok{sep=}\StringTok{""}\NormalTok{)}

\CommentTok{\# Read and import csv}
\NormalTok{broadcast }\OtherTok{\textless{}{-}} \FunctionTok{read.csv}\NormalTok{(read\_csv, }\AttributeTok{header=}\NormalTok{T, }\AttributeTok{dec=}\StringTok{\textquotesingle{}.\textquotesingle{}}\NormalTok{, }\AttributeTok{sep=}\StringTok{\textquotesingle{},\textquotesingle{}}\NormalTok{, }\AttributeTok{na.strings =} \StringTok{""}\NormalTok{)}

\FunctionTok{numSummary}\NormalTok{(broadcast[,}\FunctionTok{c}\NormalTok{(}\StringTok{"COM\_TIME"}\NormalTok{), }\AttributeTok{drop=}\ConstantTok{FALSE}\NormalTok{], }\AttributeTok{groups =}\NormalTok{ broadcast}\SpecialCharTok{$}\NormalTok{NPROC, }\AttributeTok{statistics=}\FunctionTok{c}\NormalTok{(}\StringTok{"mean"}\NormalTok{, }\StringTok{"sd"}\NormalTok{, }\StringTok{"IQR"}\NormalTok{, }\StringTok{"quantiles"}\NormalTok{, }\StringTok{"skewness"}\NormalTok{, }\StringTok{"kurtosis"}\NormalTok{), }\AttributeTok{quantiles=}\FunctionTok{c}\NormalTok{(}\DecValTok{0}\NormalTok{,.}\DecValTok{25}\NormalTok{,.}\DecValTok{5}\NormalTok{,.}\DecValTok{75}\NormalTok{,}\DecValTok{1}\NormalTok{))}
\end{Highlighting}
\end{Shaded}

\begin{verbatim}
##            mean           sd        IQR    skewness    kurtosis       0%
## 4  0.0003353750 0.0001943428 0.00028575 -0.09104056 -1.07480131 0.000027
## 8  0.0009428125 0.0005842683 0.00059750  1.31355553  0.22483039 0.000506
## 16 0.0018172500 0.0015464069 0.00182450  1.17230684  0.09561119 0.000386
##           25%      50%        75%     100% COM_TIME:n
## 4  0.00024675 0.000277 0.00053250 0.000567          8
## 8  0.00056275 0.000640 0.00116025 0.002171         16
## 16 0.00058275 0.001206 0.00240725 0.005297         32
\end{verbatim}

\begin{Shaded}
\begin{Highlighting}[]
\CommentTok{\# Grafía de Boxplot de Homicidios por regiones}
\FunctionTok{Boxplot}\NormalTok{(COM\_TIME}\SpecialCharTok{\textasciitilde{}}\NormalTok{NPROC, }\AttributeTok{data=}\NormalTok{broadcast, }\AttributeTok{id=}\FunctionTok{list}\NormalTok{(}\AttributeTok{method=}\StringTok{"y"}\NormalTok{))}
\end{Highlighting}
\end{Shaded}

\includegraphics{informe_files/figure-latex/unnamed-chunk-3-1.pdf}

\begin{Shaded}
\begin{Highlighting}[]
\FunctionTok{plot}\NormalTok{(cars)}
\end{Highlighting}
\end{Shaded}

\includegraphics{informe_files/figure-latex/unnamed-chunk-4-1.pdf}

\end{document}
